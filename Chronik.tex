% Options for packages loaded elsewhere
\PassOptionsToPackage{unicode}{hyperref}
\PassOptionsToPackage{hyphens}{url}
\documentclass[
]{article}
\usepackage{xcolor}
\usepackage{amsmath,amssymb}
\setcounter{secnumdepth}{-\maxdimen} % remove section numbering
\usepackage{iftex}
\ifPDFTeX
  \usepackage[T1]{fontenc}
  \usepackage[utf8]{inputenc}
  \usepackage{textcomp} % provide euro and other symbols
\else % if luatex or xetex
  \usepackage{unicode-math} % this also loads fontspec
  \defaultfontfeatures{Scale=MatchLowercase}
  \defaultfontfeatures[\rmfamily]{Ligatures=TeX,Scale=1}
\fi
\usepackage{lmodern}
\ifPDFTeX\else
  % xetex/luatex font selection
\fi
% Use upquote if available, for straight quotes in verbatim environments
\IfFileExists{upquote.sty}{\usepackage{upquote}}{}
\IfFileExists{microtype.sty}{% use microtype if available
  \usepackage[]{microtype}
  \UseMicrotypeSet[protrusion]{basicmath} % disable protrusion for tt fonts
}{}
\makeatletter
\@ifundefined{KOMAClassName}{% if non-KOMA class
  \IfFileExists{parskip.sty}{%
    \usepackage{parskip}
  }{% else
    \setlength{\parindent}{0pt}
    \setlength{\parskip}{6pt plus 2pt minus 1pt}}
}{% if KOMA class
  \KOMAoptions{parskip=half}}
\makeatother
\usepackage{longtable,booktabs,array}
\usepackage{calc} % for calculating minipage widths
% Correct order of tables after \paragraph or \subparagraph
\usepackage{etoolbox}
\makeatletter
\patchcmd\longtable{\par}{\if@noskipsec\mbox{}\fi\par}{}{}
\makeatother
% Allow footnotes in longtable head/foot
\IfFileExists{footnotehyper.sty}{\usepackage{footnotehyper}}{\usepackage{footnote}}
\makesavenoteenv{longtable}
\setlength{\emergencystretch}{3em} % prevent overfull lines
\providecommand{\tightlist}{%
  \setlength{\itemsep}{0pt}\setlength{\parskip}{0pt}}
\usepackage{bookmark}
\IfFileExists{xurl.sty}{\usepackage{xurl}}{} % add URL line breaks if available
\urlstyle{same}
\hypersetup{
  hidelinks,
  pdfcreator={LaTeX via pandoc}}

\author{}
\date{}

\begin{document}

\section{Chronik der Familie Fleschutz}\label{header-n0}

Download als: \href{Chronik.docx}{.DOCX} \textbar{}
\href{Chronik.epub}{E-Book} \textbar{} \href{Chronik.html}{.HTML}
\textbar{} \href{Chronik.odt}{.ODT} \textbar{}
\href{Chronik.opml}{.OPML} \textbar{} \href{Chronik.pdf}{.PDF}
\textbar{} \href{Chronik.rtf}{.RTF}

\begin{longtable}[]{@{}ll@{}}
\toprule\noalign{}
Jahr & Ereignis (* = Geburt, \& = Hochzeit, + = Tod, {[}{]} = Beruf,
\{\} = Quelle) \\
\midrule\noalign{}
\endhead
\bottomrule\noalign{}
\endlastfoot
1412 & Utz Brästel genannt Fläschüzen, kauft die Güter zu "Wyler" und
"Mätzlins" (jetzt Fleschützen bei Börwang) vom Fürststift Kempten
\{\href{Quellen/Fuerststift_Kempten/Urkunde_263/}{Urkunde 263}\}, erste
urkundliche Erwähnung von Fleschutz \\
1480 & Jörg Fleschütz, Haldenwanger Pfarr. \\
1505 & Vergleich zwischen Hans Johann Fleschutz und Hans Caspar
Laubenberg \{\href{Quellen/Fuerststift_Kempten/Urkunde_1757/}{Urkunde
1757}\} \\
1516 & Frevelgerichtsbarkeit zu Fleschützen
\{\href{Quellen/Fuerststift_Kempten/Urkunde_2007/}{Urkunde 2007}\} \\
1525 & Deutscher Bauernkrieg, 200 Allgäuer Höfe werden in Brand gesteckt
\{\href{Quellen/Wikipedia/Deutscher_Bauernkrieg/}{Wikipedia}\} \\
1530 & Georg Fleschutz (Hofmeister) kauft Wasserrecht zu Burkarts
\{\href{Quellen/Fuerststift_Kempten/Urkunde_2546/}{Urkunde 2546}\} \\
1540 & Georg Fleschutz (Hofmeister) kauft Haus vom Konvent
\{\href{Quellen/Fuerststift_Kempten/Urkunde_2915/}{Urkunde 2915}\} \\
1542 & Georg Fleschutz (Hofmeister) kauft 2 Häuser vom Konvent
\{\href{Quellen/Fuerststift_Kempten/Urkunde_2984}{Urkunde 2984}\} \\
1543 & Baltus Fleschutz, zum Weyler, genannt (bei den) Fleschutzen \\
1544 & Georg Fleschutz, Hofmeister im Stift Kempten \\
1550 & Agatha Fleschutz verkauft ihr Gut zu Eschers (Untrasried) für 200
Gulden \{\href{Quellen/Fuerststift_Kempten/Urkunde_3316}{Urkunde
3316}\} \\
1550-1743 & Güter und Untertanen zu Fleschützen
\{\href{Quellen/Fuerststift_Kempten/Akte_1913}{Akte 1913}\} \\
1554 & Georg Fleschutz (in Schwarzen) wegen verliehener
Wirtschaftsgerechtsame
\{\href{Quellen/Fuerststift_Kempten/Urkunde_3495/}{Urkunde 3495}\} und
\{\href{Quellen/Fuerststift_Kempten/Urkunde_3499}{Urkunde 3499}\} \\
1564 & Baltasar Fleschutz (Scholare) will Priester werden \\
1565 & Christoph Fleschutz kauft ein Haus mit Taferngerechtigkeit
\{\href{Quellen/Fuerststift_Kempten/Urkunde_3800}{Urkunde 3800}\} \\
1565 & Balthus Fleschutz zu Fleschützen bekommt Zinsbrief von Lukas
Haini zu Bachtels
\{\href{Quellen/Fuerststift_Kempten/Urkunde_3778}{Urkunde 3778}\} \\
1618-1648 & Dreißigjähriger Krieg, dadurch Hungersnöte und Seuchen. In
Teilen Süddeutschlands überlebte nur ein Drittel der Bevölkerung
\{\href{Quellen/Wikipedia/Dreissigjaehriger_Krieg}{Wikipedia}\} \\
1658 & Hans Georg Fleschutz verkauft Baind zu Dickenbühl
\{\href{Quellen/Fuerststift_Kempten/Urkunde_5642/}{Urkunde 5642}\} \\
1666 & Baltasar Fleschutz, Bauschreiber im Stift Kempten \\
1686 & Georg Fleschutz zu Haubensteig kauft Weiderecht im Stadtallmey
\{\href{Quellen/Fuerststift_Kempten/Urkunde_1127/}{Urkunde 1127}\} \\
\end{longtable}

\begin{longtable}[]{@{}ll@{}}
\toprule\noalign{}
Vorname(n) & Ereignis (* = Geburt, \& = Hochzeit, + = Tod, {[}{]} =
Beruf, \{\} = Quelle) \\
\midrule\noalign{}
\endhead
\bottomrule\noalign{}
\endlastfoot
Baltasar & \& Ursula Schneider, +1646 (vermutlich verwandt)
\{\href{https://data.matricula-online.eu/de/deutschland/augsburg/haldenwang-bei-kempten/1-S/?pg=1}{Sterberegister}\} \\
& \\
& \textbf{Kind von Baltasar:} \\
\textbf{Georg} & \&1649 mit Sabina Brenberg(+1652) \&1654 mit Maria
Heslin in Börwang {[}Hauptmann und Wirt{]} (vermutlich verwandt)
\{\href{https://data.matricula-online.eu/de/deutschland/augsburg/haldenwang-bei-kempten/1-H/?pg=11}{Hochzeitsregister}\} \\
& \\
& \textbf{Kinder von Georg \& Sabina / Maria:} \\
\textbf{Johann Georg} & \&1663 mit Maria Briechler in Börwang
\{\href{https://data.matricula-online.eu/de/deutschland/augsburg/haldenwang-bei-kempten/1-H/?pg=19}{Hochzeitsregister}\} \\
Maria & *1649 in Börwang (vermutlich verwandt)
\{\href{https://data.matricula-online.eu/de/deutschland/augsburg/haldenwang-bei-kempten/1-T-1/?pg=10}{Taufregister}\} \\
Catharina & *1651 in B. +1654 (vermutlich verwandt)
\{\href{https://data.matricula-online.eu/de/deutschland/augsburg/haldenwang-bei-kempten/1-T-1/?pg=25}{Taufregister}\} \\
Ursula & *1655 in B. (vermutlich verwandt)
\{\href{https://data.matricula-online.eu/de/deutschland/augsburg/haldenwang-bei-kempten/1-T-1/?pg=41}{Taufregister}\} \\
Elisabeth & *1656 in B. (vermutlich verwandt)
\{\href{https://data.matricula-online.eu/de/deutschland/augsburg/haldenwang-bei-kempten/1-T-1/?pg=45}{Taufregister}\} \\
Regina & *1657 in B. (vermutlich verwandt)
\{\href{https://data.matricula-online.eu/de/deutschland/augsburg/haldenwang-bei-kempten/1-T-1/?pg=51}{Taufregister}\} \\
Sabina & *1659 in B. (vermutlich verwandt)
\{\href{https://data.matricula-online.eu/de/deutschland/augsburg/haldenwang-bei-kempten/1-T-1/?pg=57}{Taufregister}\} \\
& \\
& \textbf{Kinder von Johann Georg \& Maria:} \\
Sabina & *1664 in Börwang (Haldenwang) \\
Roman & *1665 in B. \\
Rosina & *1666 in B. \\
Ferdinand & *1667 in B. \\
Anna & *1669 in B. \\
Barbara & *1670 in B. \\
Franz & *1671 in B. \\
Maria & *1673 in B. \\
Maria & *1675 in B. \\
Franz & *1677 in B. \\
Anna & *1679 in B. \\
\textbf{Mang} & *1680 in B., \&1717 Anna Maria Geiger in B. \\
Josef & *1681 in B. \\
Georg & *1683 in B. \\
& \\
& \textbf{Kinder von Mang \& Anna Maria:} \\
Anna & *1718 in Börwang (Haldenwang)
\{\href{https://data.matricula-online.eu/de/deutschland/augsburg/haldenwang-bei-kempten/3-T/?pg=34}{Taufregister}\} \\
Anna Maria & *1719 in B. \\
Ferdinand & *1721 in B. \\
Johannes? & *1723 in B. \\
\textbf{Josef} & *1725 in B., \&11.08.1757 Anastasia Trunzer in
Untrasried, +06.04.1773 in Waizenried (Untrasried) \\
Dominik & *1726 in B. \\
& \\
& \textbf{Kinder von Josef \& Anastasia:} \\
Anna Barbara & *23.03.1758 in Weizenried 79 bei Untrasried (jetzt
Schindele-Hof) \&23.05.1793 mit Prack
\{\href{https://data.matricula-online.eu/de/deutschland/augsburg/untrasried/16-FB/?pg=99}{Familienbuch}\} \\
Jakob & *23.07.1760 in W., +01.08.1760 mit nur 9 Tagen \\
\textbf{Leonhard} & *04.11.1761 in W, \&11.7.1785 mit Maria Adelheid
Waldmann, +09.03.1814 {[}Bauer{]} \\
Dominik & *08.10.1762 in W. +1762 \\
Maria Anna & *18.07.1764 in W. \&04.02.1788 mit Stehele \ldots{} nach
Mittelberg \\
Sabina & *26.10.1765 in W. +04.01.1766 mit nur 2 Monaten \\
Jakob & *06.07.1768 in W. \\
Afra & *06.08.1769 in W. +10.09.1769 mit nur 1 Monat \\
Theodor & *08.11.1770 \\
Anna Maria & *31.12.1772 in W. +18.08.1773 mit nur 7 Monaten \\
- & \textquotesingle+23.10.1773 (notgetauft) \\
Josef & *16.02.1775 in W. +09.03.1775 mit nur 21 Tagen \\
Franz Josef & *14.02.1776 \\
& \\
& \textbf{Kinder von Leonhard \& Maria:} \\
Johann Georg & *07.07.1786 in Weizenried 79 bei Untrasried, +13.07.1786
\{\href{https://data.matricula-online.eu/de/deutschland/augsburg/untrasried/16-FB/?pg=99}{Familienbuch}\} \\
Anna Maria & *29.08.1787 in W. +04.09.1787 (nur 6 Tage) \\
- & \textquotesingle+24.09.1788 (notgetauft) \\
Genovefa & *03.01.1790 in W. +09.01.1790 (nur 6 Tage) \\
\textbf{Johann Georg} & *20.04.1791 in W., \& mit Maurus ... \& mit
Kreszentia Reichart, +06.06.1865 {[}Bauer{]} \\
M. Afra & *05.08.1794 in W. \\
Ulrich & *04.07.1796 in W. +17.09.1861 in Burg \& mit Creszentia
Hartmann, *25.12.1791 \\
Franziska & *03.10.1797 in W. \\
Johann Baptist & *23.06.1799 in W., +13.02.1875 in Engetried \& Maria
Kreszenz Epp *10.09.1798 +06.12.1862 \\
Maria Anna & *16.07.1805 in W., +1.7.1810 mit nur 5 Jahren \\
& \\
& \textbf{Kinder von Johann Georg:} \\
Franz Xaver & *03.02.1818 in Weizenried 79 bei Untrasried, +06.02.1818
mit nur 3 Tagen
\{\href{https://data.matricula-online.eu/de/deutschland/augsburg/untrasried/16-FB/?pg=99}{Familienbuch}\} \\
Maria Anna & *12.10.1819 in W., +14.07.1867 in Kraftisried? \\
Karolina & *16.03.1821 in W. \\
Franz Xaver & *13.05.1822 in W. +19.05.1822 mit nur 6 Tagen \\
Johann Georg & *14.08.1823 in W. +24.04.1830 \\
Johann ? & *02.08.1824 in W. +28.08.1824 mit nur 1 Monat \\
Ignaz & *31.07.1825 in W. +17.09.1825 mit nur 46 Tagen \\
M. Josefa & *31.10.1826 in W. \\
Johannes Chrysostomus & *09.02.1828 in W. +1907 in Obg. \&24.11.1862
Maria Antonia Schindele (zog als Privatier nach Obg.) \\
Johann L. & *24.06.1829 in W. +02.03.1830 \\
\textbf{Theresia} & *01.06.1831 in W. +25.11.1901 in Ostenried 71
(Untrasried) {[}Privatiere{]} \{Sterbebild\} \\
Theodor & *20.10.1832 in W. +1915 in Albrechts \\
Alois & *24.03.1834 \\
Johann Georg & *19.11.1835 in W. +03.04.1880 in Ostenried 71 \\
Johann Heinrich & *27.04.1837 in W. \&21.2.1881 in Altdorf mit Maria
Anna T. (2 Monate Hof, Trübsinn) \\
& \\
& \textbf{Kind von Theresia}: \\
\textbf{Johann Georg} & *09.05.1868 in Ostenried 71 bei Untrasried
+05.01.1933 \& Apollonia Mayr *09.02.1870 +08.12.1957 {[}Bauer{]} \\
& \\
& \textbf{Kinder von Johann Georg \& Apollonia:} \\
\textbf{Johann} & *30.12.1895 in Ostenried +29.05.1955 in Albrechts \&
Sophie Hartmann *23.03.1904 +30.09.1977 {[}Bauer{]} \\
Maria & *25.01.1897 in Ostenried +05.01.1990 \\
Theresia & *27.04.1902 in Ostenried +25.06.1987 \& Johann Kustermann \\
Georg & *19.04.1903 in Ostenried +19.04.1903 \\
Johann Georg & *13.08.1906 in Albrechts +09.05.1935 \\
Theodor & *10.12.1907 in Albrechts +28.09.1942 bei Leningrad, Russland
\{Sterbebild\} \\
& \\
& \textbf{Kinder von Johann \& Sophie:} \\
Georg & *21.01.1935 in Albrechts 12 (Günzach) +19.03.1935 \\
Amalie Maria Anna & *20.02.1936 in A. \\
Apollonia Theresia & *29.05.1937 in A. \\
Johann & *05.12.1938 in A. \& Rosmarie Höbel *18.12.1947 {[}Bauer{]} \\
Theodor Konrad & *12.11.1942 in A. \& Sigrun Friede *01.04.1949 in
Radolfzell {[}Molkerei-Meister{]} \\
\end{longtable}

\subsection{Danksagung}\label{header-n369}

Vielen Dank an Karl Fleschutz und an seinen Vater für ihre wertvolle
Chronik. Vielen Dank an Bernhard für die Sterbebilder und an Jörg für
den Hinweis zu Matricula Online.

\end{document}
